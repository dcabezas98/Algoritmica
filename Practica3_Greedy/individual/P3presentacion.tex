\documentclass[serif,9pt]{beamer}
\usetheme{tree}

\usepackage[utf8]{inputenc}
\usepackage[spanish]{babel}
\usepackage{caption}
\usepackage{graphicx} % figuras
\usepackage{subfigure} % subfiguras
\usepackage{float}

\usepackage{amsmath,amssymb,amsthm}

\usepackage{multirow}

\usepackage{changepage}

\restylefloat{figure}

\beamersetuncovermixins{\opaqueness<1>{25}}{\opaqueness<2->{15}}

\AtBeginSection[]
{
  \begin{frame}<beamer>{Contenido}
    \tableofcontents[currentsection]
  \end{frame}
}

\setcounter{section}{-1}

\begin{document}

\title{Práctica 3: Algoritmos Greedy}  
\author{David Cabezas Berrido}
\date{}

\begin{frame}
\titlepage
\end{frame}

\begin{frame}{Contenido}
\tableofcontents
\end{frame} 

\section{El problema}

\begin{frame}{El problema}
Lista de reparaciones, con distintos tiempos
\[T=[t_1,\ldots,t_n]\]
Debemos minimizar el tiempo medio de espera de los clientes.
\end{frame}

\section{Caso de un único electricista}

\subsection{Enfoque greedy}

\begin{frame}{Enfoque greedy}
\begin{enumerate}
\item \textbf{Conjunto de candidatos:} Reparaciones a efectuar.
  \pause
\item \textbf{Candidatos ya usados:} Reparaciones ya efectuadas.
  \pause
\item \textbf{Criterio solución:} Una lista con todas las reparaciones
  es solución.
  \pause
\item \textbf{Criterio factible:} Cualquier lista de reparaciones
  puede llegar a ser solución.
  \pause
\item \textbf{Función de selección:} A determinar por el algoritmo.
  \pause
\item \textbf{Función objetivo:} A una lista solución le asocia la
  suma de los tiempos de espera de los clientes (debemos minimizarla).
\end{enumerate}
\end{frame}

\subsection{Deducción del algoritmo}

\begin{frame}{Deducción del algoritmo}
  Realizamos las tareas en orden de subíndice ($t_1,t_2,\ldots,t_n$).\\
  \pause
  $C_i$, tiempo de espera del cliente $i$-ésimo.
  \[C_i=\sum_{j=1}^it_j\]
  \pause
  $C$, suma de los tiempos de espera de los clientes.
  \[C = \sum_{i=1}^n C_i = \sum_{i=1}^n \sum_{j=1}^i t_j =
    \sum_{i=1}^n(n+1-i)t_i\]
  \pause
  Solución: Realizar las reparaciones de menor a mayor tiempo
  requerido.
\end{frame}

\subsection{Demostración}

\begin{frame}{Demostración}
  Por reducción al absurdo:
  
  \vspace{5mm}
  
  Sea $S=[s_1,s_2\ldots,s_n]$ la
  permutación de $T$ que presenta la solución óptima. \\ \pause
  Lleguemos a un absurdo suponiendo que $S$ no cumple
  $s_1 \leq s_2 \leq \ldots \leq s_n$, esto es:
  \[\exists p,q \in \{1,2,\ldots,n\} \mbox{ tales que } p<q \mbox{ y }
    s_p > s_q\]
  \pause
  \vspace{-5mm}
  \begin{align*}
    s_p > s_q& \qquad n+1-q < n+1-p\\
    s_p > s_q& \qquad ((n+1-p) - (n+q-q)) > 0 \\
  \end{align*}
  \pause
  \vspace{-10mm}
  \begin{align*}    
    ((n+1-p) - (n+1-q))s_p > ((n+1-p) - (n+1-q))s_q \\
    (n+1-p)s_p - (n+1-q)s_p > (n+1-p)s_q - (n+1-q)s_q
  \end{align*}
  \pause
  y por tanto
  \[(n+1-p)s_p + (n+1-q)s_q > (n+1-p)s_q + (n+1-q)s_p\]
  lo que contradice a que $S$ sea la solución óptima. \\
  \hfill\qedsymbol
\end{frame}

\section{Caso de varios electricistas}

\subsection{Deducción del algoritmo}

\begin{frame}{Planteamiento del problema}
  Dividimos las tareas de $T$ en $p$ listas
  \[T_k=[t^k_1,\ldots,t^k_{n_k}] \quad k=1,\ldots,p\]
  \pause
  Debemos minimizar
  \[C=\sum_{k=1}^pC_k \qquad C_k=\sum_{i=1}^{n_k}(n_k+1-i)t_i^k \qquad
    n=\sum_{k=1}^pn_k\]
  \pause
  Tendremos por lo anterior
  \[t_1^k \leq t_2^k \leq \ldots \leq t_{n_k}^k\]
  Queda razonar la mejor forma de repartir las tareas.
\end{frame}

\begin{frame}{Razonamiento}
  Coeficientes de la sumatoria:
  \begin{align*}
    1,2,\ldots,n_1 \quad& (\text{electricista }1)\\
    1,2,\ldots,n_2 \quad& (\text{electricista }2)\\
    \vdots& \\
    1,2,\ldots,n_k \quad& (\text{electricista }k)\\
    \vdots& \\
    1,2,\ldots,n_p \quad& (\text{electricista }p)
  \end{align*}
\end{frame}

\subsection{Implementación}

\begin{frame}[fragile]{Código}
\begin{verbatim}
    1  int optimo(vector<int> T){
    2    
    3    sort(T.begin(),T.end());
    4  
    5    vector<vector<int>> electricistas(p);
    6    for(int i = 0; i < T.size(); i++)
    7      electricistas[i%p].push_back(T[i]);
    8  
    9    int suma = 0;
   10  
   11    for(vector<int> e: electricistas)
   12      suma += sumaTiempoEspera(e);
   13  
   14    return suma;
   15  }
\end{verbatim}
\end{frame}

\subsection{Ejemplo}

\begin{frame}{Ejemplo}
  $T=[2, 8, 5, 9, 2, 6, 15]$, 3 electricistas. \\
  \pause
  \vspace{5mm}
  Ordenamos las reparaciones de menor a mayor tiempo requerido:
  \[T'=[2, 2, 5, 6, 8, 9, 15]\]
  \pause
  Repartimos el trabajo entre los electricistas:
  \begin{align*}
    &T_1=[2,6,15] \\
    &T_2=[2,8] \\
    &T_3=[5,9]
  \end{align*}
  \pause
  Suma de los tiempos de espera:
  \[C=C_1+C_2+C_3=(2+8+23)+(2+10)+(5+14)=33+12+19=64\]
\end{frame}

\subsection{Simulaciones}

\begin{frame}{Simulaciones}
  Formato: Suma de los tiempos de espera del algoritmo/Mejor suma de
  tiempos de espera de 100 formas aleatorias de distribuir las tareas.
\begin{table}[H]
\centering
\caption*{Suma de los tiempos de espera en las distintas simulaciones}
\label{tabla:simulaciones}
\begin{adjustwidth}{-11.5mm}{}
\begin{tabular}{|c|c|c|c|c|c|c|}
\hline
\multicolumn{2}{|c|}{\multirow{2}{*}{}} & \multicolumn{5}{c|}{n}     \\ \cline{3-7} 
\multicolumn{2}{|c|}{}                  & 50 & 100 & 150 & 200 & 300 \\ \hline
\multirow{4}{*}{p}         & 5          & 12105/12232 & 44294/44885 & 98028/98746 & 166911/167778 & 365856/367299 \\ \cline{2-7} 
                           & 10         & 5287/5635 & 22536/23448 & 52148/53540 & 81574/83035 & 182824/185350 \\ \cline{2-7} 
                           & 25         & 3520/4117 & 11905/13137 & 22167/24154 & 38595/41055 & 77201/81802 \\ \cline{2-7} 
                           & 40         & 3120/3618 & 8007/9297 & 16855/19140 & 26263/29593 & 52326/57121 \\ \hline
\end{tabular}
\end{adjustwidth}
\end{table}
\end{frame}

\end{document}