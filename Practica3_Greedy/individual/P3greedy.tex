\documentclass[a4]{article}

\usepackage[left=3cm,right=3cm,top=2cm,bottom=2cm]{geometry} 

\usepackage[utf8]{inputenc} 
\usepackage[           spanish % para poder usar el español
                      ,es-tabla % para los captions de las tablas
                       ]{babel}   
\decimalpoint

\usepackage[bookmarks=true,
            bookmarksnumbered=false, % true means bookmarks in 
                                     % left window are numbered
            bookmarksopen=false,     % true means only level 1
                                     % are displayed.
            colorlinks=true,
            linkcolor=blue,
            urlcolor=cyan]{hyperref}
            
\usepackage[T1]{fontenc}
\usepackage{lmodern}

\usepackage{parskip}
\usepackage{xcolor}

\usepackage{multirow}

\usepackage{amsmath,amssymb,amsthm}

\usepackage{caption}

\usepackage{listings}
\lstset
{ %Formatting for code in appendix
  language=C++, % choose the language of the code
  basicstyle=\fontfamily{pcr}\selectfont\footnotesize\color{black},
  keywordstyle=\color{darkorange}\bfseries, % style for keywords
  numbers=left, % where to put the line-numbers
  numberstyle=\tiny, % the size of the fonts that are used for the line-numbers     
  backgroundcolor=\color{white},
  showspaces=false, % show spaces adding particular underscores
  showstringspaces=false, % underline spaces within strings
  showtabs=false, % show tabs within strings adding particular underscores
  tabsize=2, % sets default tabsize to 2 spaces
  captionpos=b, % sets the caption-position to bottom
  breaklines=true, % sets automatic line breaking
  breakatwhitespace=false, 
}

\usepackage{enumerate}% paquete para poder personalizar fácilmente la apariencia de las listas enumerativas

\usepackage{graphicx} % figuras
\usepackage{subfigure} % subfiguras

\definecolor{darkorange}{rgb}{0.94,0.4,0.0}
	
\usepackage{float} % para controlar la situación de los entornos flotantes

\restylefloat{figure}
\restylefloat{table} 

\newcommand{\HRule}{\rule{\linewidth}{0.5mm}}

\author{David Cabezas Berrido}
\date{\vspace{-5mm}}

\title{\huge Práctica 3: Algoritmos Greedy \HRule\vspace{-4mm}}

\setcounter{section}{-1}

\begin{document}
\maketitle
\vspace{20mm}
\tableofcontents
\newpage

\section{Problema de las reparaciones}
Un electricista necesita hacer $n$ reparaciones, se sabe de antemano
que la tarea $i$-ésima tardará $t_i$ minutos. El objetivo es minimizar
el tiempo medio de espera de los clientes. Debemos diseñar un
algoritmo que tome una lista de tiempos de tareas como entrada y
devuelva la permutación de la misma en la que el tiempo medio de
espera de los clientes sea óptimo.

Lo haremos también para el caso de varios electricistas.

\section{Caso de un único electricista}

\subsection{Enfoque greedy}
\begin{enumerate}
\item \textbf{Conjunto de candidatos:} Reparaciones a efectuar.
\item \textbf{Candidatos ya usados:} Reparaciones ya efectuadas.
\item \textbf{Criterio solución:} Una lista con todas las reparaciones
  es solución.
\item \textbf{Criterio factible:} Cualquier lista de reparaciones
  puede llegar a ser solución.
\item \textbf{Función de selección:} A determinar por el algoritmo.
\item \textbf{Función objetivo:} A una lista solución le asocia la
  suma de los tiempos de espera de los clientes (debemos minimizarla).
\end{enumerate}

Debemos determinar el orden de realización de las reparaciones que
presente la menor suma de tiempos de espera de los clientes. Es decir,
debemos encontrar un criterio para elegir que candidato pasar a la
lista solución con el objetivo de que la suma de los tiempos de espera
sea mínima.

\subsection{Deducción del algoritmo}
Las tareas se realizarán en el orden indicado por su subíndice:
$[t_1, t_2,\ldots,t_n]$.

Notaremos $C_i$ al tiempo de espera del cliente $i$-ésimo. Este tendrá
que esperar a que se realicen tanto su reparación como todas las
anteriores, el primer cliente sólo tendrá que esperar el tiempo que
tome su reparación ($C_1=t_1$), el segundo el tiempo de la primera
reparación y el de la suya ($C_2=t_1+t_2$). En general:
\[C_i=\sum_{j=1}^it_j \qquad \forall i \in \{1,2,\ldots,n\}\]
Nuestro objetivo es minimizar el tiempo medio de espera de cada
cliente, por lo tanto la suma de los $C_i$ debe ser lo más baja
posible. Llamemos $C$ a esta suma:
\[C = \sum_{i=1}^n C_i = \sum_{i=1}^n \sum_{j=1}^i t_j =
  \sum_{i=1}^n(n+1-i)t_i\]
Como observamos, los tiempos de las tareas quedan multiplicados por
coeficientes cada vez menores. Por tanto, el tiempo total de espera es
menor cuando las reparaciones que requieren menos tiempo se realizan
antes. Por ello, propongo como algoritmo realizar las tareas en orden
de menor a mayor tiempo requerido.

\newpage
\subsection{Demostración de que produce la solución óptima}

Para probar que este algoritmo produce la salida óptima, lo he hecho
de dos formas: por inducción y por reducción al absurdo.

\subsubsection{Inducción sobre el número de reparaciones}

En el caso $n=1$, hay una única solución y por tanto óptima.

Supongamos que nuestro algoritmo siempre encuentra la solución óptima
en un problema de tamaño $n$, y consideremos un problema con una
lista $T$ de $n+1$ elementos con los tiempos de $n+1$ reparaciones
urgentes.  $T=[t_1,t_2,\ldots,t_n,t_{n+1}]$.

Notaremos $R=[r_1,r_2,\ldots,r_n,r_{n+1}]$ a la solución que produce
nuestro algorítmo y $S=[s_1,s_2,\ldots,s_n,s_{n+1}]$ a una solución
óptima (existe porque el número de posibles soluciones es finito,
$n!$). Ambas son permutaciones de $T$. Tenemos: \vspace{-4mm}
\[C_r = \sum_{i=1}^{n+1} (n+2-i)r_i \qquad C_s = \sum_{i=1}^{n+1} (n+2-i)s_i\]
Sabemos que $C_s \leq C_r$, ya que $S$ es la solución
óptima. Debemos probar que se da la igualdad.

Tomemos el primer sumando de cada sumatoria
\[C_r = (n+1)r_1 + C_r^* \qquad C_r^* = \sum_{i=2}^{n+1}(n+2-i)r_i\]
\[C_s = (n+1)s_1 + C_s^* \qquad C_s^* = \sum_{i=2}^{n+1}(n+2-i)s_i\]
Como nuestro algoritmo prioriza siempre la reparación que menos tiempo
requiera, podemos asegurar $r_1 \leq s_1, \ (n+1)r_1 \leq
(n+1)s_1$. Luego como sabíamos que $C_s \leq C_r$, deducimos que
$C_s^* \leq C_r^*$.

Consideramos ahora los problemas dados por
$R^*=[r_2,r_3,\ldots,r_n,r_{n+1}]$ y
$S^*=[s_2,s_3,\ldots,s_n,s_{n+1}]$, ambos de tamaño $n$. Sabemos que
nuestro algoritmo produce para $R^*$ la solución óptima, con $C^*_r$
como suma de los tiempos de espera ya que
\[\sum_{i=1}^n(n+1-i)r_{i+1} = \sum_{i=2}^{n+1}(n+2-i)r_i = C^*_r\]
Si $r_1=s_1$, $R^*$ y $S^*$ serían el mismo problema, por lo que
$C_s^*$ sería una suma de tiempos para alguna solución del problema
dado por $R^*$, para el que $C_r^*$ es mínima. Pero sabíamos
previamente que $C_s^* \leq C_r^*$, luego se dará la igualdad como
queríamos.
\[C_s^* = C_r^*, \ (n+1)r_1=(n+1)s_1 \implies C_s = C_r\]

Probemos que esta es la única posibilidad llegando a un absurdo
partiendo de $r_1 \neq s_1$.

En este caso tendremos $r_1 < s_1$, extraemos el
tiempo $r_1$ de la lista $S$, supondremos que se corresponde con $s_k$
con $k > 1$. Tendremos entonces
\begin{align*}
  s_1 > r_1 = s_k& \qquad n+2-k < n+1 \\
  s_1 > s_k& \qquad n+1 - (n+2-k) > 0 \\
\end{align*}

\vspace{-15mm}

\begin{align*}    
  (n+1-(n+2-k))s_1 > (n+1-(n+2-k))s_k \\
  (n+1)s_1 - (n+2-k)s_1 > (n+1)s_k - (n+2-k)s_k
\end{align*}
y por tanto
\[(n+1)s_1 + (n+2-k)s_k > (n+1)s_k + (n+2-k)s_1\]

Esto significa que intercambiar las posiciones de los tiempos $s_1$ y
$s_k$ en la lista $S$ produciría una suma de tiempos menor a la
existente, lo que contradice la hipótesis de que $S$ es una solución
óptima.

\hfill\qedsymbol

\vspace{-5mm}

\subsubsection{Reducción al absurdo}

Supongamos que un problema de reparaciones dado por la lista
$T=[t_1,t_2,\ldots,t_n]$, sean $R=[r_1,r_2\ldots,r_n]$ y
$S=[s_1,s_2\ldots,s_n]$ dos permutaciones de $T$. Donde $R$ es la
salida que produciría nuestro algoritmo y $S$ es la solución óptima.
Sabemos que $R$ cumple $r_1 \leq r_2 \leq \ldots \leq r_n$, probemos
que $S$ debe cumplir lo mismo.

La suma de los tiempos de espera de $S$ es menor o igual que la de
cualquier otra solución. Por tanto \(C_s = \sum_{i=1}^n (n+1-i)s_i\)
es mínima. Supongamos que $S$ no estuviera ordenada en orden no
decreciente, esto es
\[\exists p,q \in \{1,2,\ldots,n\} \mbox{ tales que } p<q \mbox{ y } s_p
  > s_q\]
Tendremos entonces
\begin{align*}
  s_p > s_q& \qquad n+1-q < n+1-p\\
  s_p > s_q& \qquad ((n+1-p) - (n+q-q)) > 0 \\
\end{align*}

\vspace{-15mm}

\begin{align*}    
  ((n+1-p) - (n+1-q))s_p > ((n+1-p) - (n+1-q))s_q \\
  (n+1-p)s_p - (n+1-q)s_p > (n+1-p)s_q - (n+1-q)s_q
\end{align*}
y por tanto
\[(n+1-p)s_p + (n+1-q)s_q > (n+1-p)s_q + (n+1-q)s_p\]

Esto significa que intercambiar las posiciones de los tiempos $s_p$ y
$s_q$ en la lista $S$ produciría una suma de tiempos menor a la
existente, lo que contradice la hipótesis de que $S$ es una solución
óptima. Tenemos entonces $s_1 \leq s_2 \leq \ldots \leq s_n$, luego
$S=R$ como queríamos.

\hfill\qedsymbol

\section{Caso de varios electricistas}

\subsection{Deducción del algoritmo}

Supongamos que tenemos $p$ electricistas para repartirse las $n$
reparaciones. Una vez que las distribuyamos, tendremos $p$ problemas
de un único electricista y cuya suma de sus tamaños es $n$. En este
caso, la suma de los tiempos de espera del problema inicial será la
suma de los tiempos de espera de cada uno de los subproblemas, por lo
que lo óptimo es que cada electricista priorice la tarea más corta de
las que tenga asignadas.

Llamaremos $T=[t_1,\ldots,t_n]$ al problema y
$\{T_k=[t^k_1,\ldots,t^k_{n_k}]\}_{k=1}^p$ serán las listas
correspondientes a los $p$ subproblemas de un electricista. Llamando
$C$ al tiempo de espera total y $C_k$ ($k=1,\ldots,p$) al tiempo de
espera de cada uno de los subproblemas obtenemos
\[C=\sum_{k=1}^pC_k \qquad C_k=\sum_{i=1}^{n_k}(n_k+1-i)t_i^k \qquad
  n=\sum_{k=1}^pn_k\]
Por lo que hemos razonado antes, en cada sublista $T_k$ tendremos
$t_1^k \leq t_2^k \leq \ldots \leq t_{n_k}^k$.

Lo único que nos queda es razonar cuál es la mejor manera de repartir
las tareas entre los electricistas para minimizar $C$.

\newpage
Antes hemos visto que la suma es mínima cuando los tiempos menores van
multiplicados por los menores coeficientes. La lista de coeficientes
de la sumatoria es:
\begin{align*}
1,2,\ldots,n_1 \quad& (\text{electricista }1)\\
1,2,\ldots,n_2 \quad& (\text{electricista }2)\\
\vdots& \\
1,2,\ldots,n_k \quad& (\text{electricista }k)\\
\vdots& \\
1,2,\ldots,n_p \quad& (\text{electricista }p)                        
\end{align*}
Por tanto, para minimizar la sumatoria debemos asignar los $p$ menores
tiempos a los $p$ electricistas (quedarán multiplicados por 1),
después asignamos las $p$ siguientes tareas más cortas una a cada
electricista (sus tiempos quedarán multiplicados por 2) y así
sucesivamente. Al final pueden quedarnos menos de $p$ tareas y que
algunos electricistas hagan menos reparaciones, por eso siempre
debemos asignar la tarea más corta de cada grupo de $p$ siempre al
mismo electricista.

\subsection{Código del algoritmo}

\begin{lstlisting}
int optimo(vector<int> T){
  
  sort(T.begin(),T.end());

  vector<vector<int>> electricistas(p);
  for(int i = 0; i < T.size(); i++)
    electricistas[i% p].push_back(T[i]);

  int suma = 0;

  for(vector<int> e: electricistas)
    suma += sumaTiempoEspera(e);

  return suma;
}
\end{lstlisting}

\subsection{Ejemplo}
$T=[2, 8, 5, 9, 2, 6, 15]$, 3 electricistas.

Ordenamos las reparaciones de menor a mayor tiempo requerido:
\[T'=[2, 2, 5, 6, 8, 9, 15]\]
Repartimos el trabajo entre los electricistas:
\begin{align*}
  &T_1=[2,6,15] \\
  &T_2=[2,8] \\
  &T_3=[5,9]
\end{align*}
Suma de los tiempos de espera:
\[C=C_1+C_2+C_3=(2+8+23)+(2+10)+(5+14)=33+12+19=64\]

\newpage
\subsection{Simulaciones}
He realizado una serie de simulaciones en las que he comparado este
algoritmo con formas aleatorias de distribuir las tareas, en siguiente
tabla se muestran los resultados.

Formato: Suma de los tiempos de espera del algoritmo/Mejor suma de tiempos de espera de 100 formas aleatorias de distribuir las tareas.

\begin{table}[H]
\centering
\caption*{Suma de los tiempos de espera en las distintas simulaciones}
\label{tabla:simulaciones}
\begin{tabular}{|c|c|c|c|c|c|c|}
\hline
\multicolumn{2}{|c|}{\multirow{2}{*}{}} & \multicolumn{5}{c|}{n}     \\ \cline{3-7} 
\multicolumn{2}{|c|}{}                  & 50 & 100 & 150 & 200 & 300 \\ \hline
\multirow{4}{*}{p}         & 5          & 12105/12232 & 44294/44885 & 98028/98746 & 166911/167778 & 365856/367299 \\ \cline{2-7} 
                           & 10         & 5287/5635 & 22536/23448 & 52148/53540 & 81574/83035 & 182824/185350 \\ \cline{2-7} 
                           & 25         & 3520/4117 & 11905/13137 & 22167/24154 & 38595/41055 & 77201/81802 \\ \cline{2-7} 
                           & 40         & 3120/3618 & 8007/9297 & 16855/19140 & 26263/29593 & 52326/57121 \\ \hline
\end{tabular}
\end{table}

En todas las simulaciones, nuestro algoritmo ha obtenido mejores
resultados que las 100 formas aleatorias de distribuir las tareas
entre los electricistas.

\end{document}