\documentclass[a4]{article}

\usepackage[left=3cm,right=3cm,top=2cm,bottom=2cm]{geometry} 

\usepackage[utf8]{inputenc}   % otra alternativa para los caracteres acentuados y la "ñ"
\usepackage[           spanish % para poder usar el español
                      ,es-tabla % para los captions de las tablas
                       ]{babel}   
\decimalpoint %para usar el punto decimal en vez de coma para los números con decimales

\usepackage[bookmarks=true,
            bookmarksnumbered=false, % true means bookmarks in 
                                     % left window are numbered              
            bookmarksopen=false,     % true means only level 1
                                     % are displayed.
            colorlinks=true,
            linkcolor=blue,
            urlcolor=cyan]{hyperref}
            
\usepackage[T1]{fontenc}
\usepackage{lmodern}

\usepackage{parskip}
\usepackage{xcolor}

\usepackage{amsmath,amssymb,amsthm}

\usepackage{caption}

\usepackage{listings}
\lstset
{ %Formatting for code in appendix
  language=C++, % choose the language of the code
  basicstyle=\fontfamily{pcr}\selectfont\footnotesize\color{black},
  keywordstyle=\color{darkorange}\bfseries, % style for keywords
  numbers=left, % where to put the line-numbers
  numberstyle=\tiny, % the size of the fonts that are used for the line-numbers     
  backgroundcolor=\color{white},
  showspaces=false, % show spaces adding particular underscores
  showstringspaces=false, % underline spaces within strings
  showtabs=false, % show tabs within strings adding particular underscores
  tabsize=2, % sets default tabsize to 2 spaces
  captionpos=b, % sets the caption-position to bottom
  breaklines=true, % sets automatic line breaking
  breakatwhitespace=false, 
}

\usepackage{enumerate}% paquete para poder personalizar fácilmente la apariencia de las listas enumerativas

\usepackage{graphicx} % figuras
\usepackage{subfigure} % subfiguras

\definecolor{darkorange}{rgb}{0.94,0.4,0.0}
	
\usepackage{float} % para controlar la situación de los entornos flotantes

\restylefloat{figure}
\restylefloat{table} 

\newcommand{\HRule}{\rule{\linewidth}{0.5mm}}

\author{David Cabezas Berrido}
\date{\vspace{-5mm}}

\title{\huge Práctica 3: Algoritmos Greedy \HRule\vspace{-4mm}}

\setcounter{section}{-1}

\begin{document}
\maketitle
\vspace{20mm}
\tableofcontents
\newpage

\section{Problema de las reparaciones}
Un electricista necesita hacer $n$ reparaciones, se sabe de antemano
que la tarea $i$-ésima tardará $t_i$ minutos. El objetivo es minimizar
el tiempo medio de espera de los clientes. Debemos diseñar un
algoritmo que tome una lista de tiempos de tareas como entrada y
devuelva la permutación de la misma en la que el tiempo medio de
espera de los clientes sea óptimo.

\section{Resolución teórica: Un único electricista}

\subsection{Deducción del algoritmo}

Las tareas se realizarán en el orden indicado por su subíndice:
$[t_1, t_2,\ldots,t_n]$

Notaremos $C_i$ al tiempo de espera del cliente $i$-ésimo. Este tendrá
que esperar a que se realicen tanto su reparación como todas las
anteriores, el primer cliente sólo tendrá que esperar el tiempo que
tome su reparación ($C_1=t_1$), el segundo el tiempo de la primera
reparación y el de la suya ($C_2=t_1+t_2$). En general:
\[C_i=\sum_{j=1}^it_j \qquad \forall i \in \{1,2,\ldots,n\}\]
Nuestro objetivo es minimizar el tiempo medio de espera de cada
cliente, por lo tanto la suma de los $C_i$ debe ser lo más baja
posible. Llamemos $C$ a esta suma:
\[C = \sum_{i=1}^n C_i = \sum_{i=1}^n \sum_{j=1}^i t_j =
  \sum_{i=1}^n(n+1-i)t_i\]
Como observamos, los tiempos de las tareas quedan multiplicados por
coeficientes cada vez menores. Por tanto, el tiempo total de espera es
menor cuando las reparaciones que requieren menos tiempo se realizan
antes. Por ello, propongo como algoritmo realizar las tareas en orden
de menor a mayor tiempo requerido.

\subsection{Demostración de que produce la solución óptima}

Para probar que este algoritmo produce la salida óptima, lo he hecho
de dos formas: por inducción y por reducción al absurdo.

\subsubsection{Inducción sobre el número de reparaciones}

En el caso $n=1$, hay una única solución y por tanto óptima.

Supongamos que nuestro algoritmo siempre encuentra la solución óptima
en un problema de tamaño $n$, y consideremos un problema con una
lista $T$ de $n+1$ elementos con los tiempos de $n+1$ reparaciones
urgentes.  $T=[t_1,t_2,\ldots,t_n,t_{n+1}]$.

Notaremos $R=[r_1,r_2,\ldots,r_n,r_{n+1}]$ a la solución que produce
nuestro algorítmo y $S=[s_1,s_2,\ldots,s_n,s_{n+1}]$ a una solución
óptima (existe porque el número de posibles soluciones es finito,
$n!$). Ambas son permutaciones de $T$. Tenemos: \vspace{-4mm}
\[C_r = \sum_{i=1}^{n+1} (n+2-i)r_i \qquad C_s = \sum_{i=1}^{n+1} (n+2-i)s_i\]
Sabemos que $C_s \leq C_r$, ya que $S$ es la solución
óptima. Debemos probar que se da la igualdad.

Tomemos el primer sumando de cada sumatoria
\[C_r = (n+1)r_1 + C_r^* \qquad C_r^* = \sum_{i=2}^{n+1}(n+2-i)r_i\]
\[C_s = (n+1)s_1 + C_s^* \qquad C_s^* = \sum_{i=2}^{n+1}(n+2-i)s_i\]
Como nuestro algoritmo prioriza siempre la reparación que menos tiempo
requiera, podemos asegurar $r_1 \leq s_1, \ (n+1)r_1 \leq
(n+1)s_1$. Luego como sabíamos que $C_s \leq C_r$, deducimos que
$C_s^* \leq C_r^*$.

Consideramos ahora los problemas dados por
$R^*=[r_2,r_3,\ldots,r_n,r_{n+1}]$ y
$S^*=[s_2,s_3,\ldots,s_n,s_{n+1}]$, ambos de tamaño $n$. Sabemos que
nuestro algoritmo produce para $R^*$ la solución óptima, con $C^*_r$
como suma de los tiempos de espera ya que
\[\sum_{i=1}^n(n+1-i)r_{i+1} = \sum_{i=2}^{n+1}(n+2-i)r_i = C^*_r\]
Si $r_1=s_1$, $R^*$ y $S^*$ serían el mismo problema, por lo que
$C_s^*$ sería una suma de tiempos para alguna solución del problema
dado por $R^*$, para el que $C_r^*$ es mínima. Pero sabíamos
previamente que $C_s^* \leq C_r^*$, luego se dará la igualdad como
queríamos.
\[C_s^* = C_r^*, \ (n+1)r_1=(n+1)s_1 \implies C_s = C_r\]

Probemos que esta es la única posibilidad llegando a un absurdo
partiendo de $r_1 \neq s_1$.

En este caso tendremos $r_1 < s_1$, extraemos el
tiempo $r_1$ de la lista $S$, supondremos que se corresponde con $s_k$
con $k > 1$. Tendremos entonces
\begin{align*}
  s_1 > r_1 = s_k& \qquad n+2-k < n+1 \\
  s_1 > s_k& \qquad n+1 - (n+2-k) > 0 \\
\end{align*}

\vspace{-15mm}

\begin{align*}    
  (n+1-(n+2-k))s_1 > (n+1-(n+2-k))s_k \\
  (n+1)s_1 - (n+2-k)s_1 > (n+1)s_k - (n+2-k)s_k
\end{align*}
y por tanto
\[(n+1)s_1 + (n+2-k)s_k > (n+1)s_k + (n+2-k)s_1\]

Esto significa que intercambiar las posiciones de los tiempos $s_1$ y
$s_k$ en la lista $S$ produciría una suma de tiempos menor a la
existente, lo que contradice la hipótesis de que $S$ es una solución
óptima.

\hfill\qedsymbol

\vspace{-5mm}

\subsubsection{Reducción al absurdo}

Supongamos que un problema de reparaciones dado por la lista
$T=[t_1,t_2,\ldots,t_n]$, sean $R=[r_1,r_2\ldots,r_n]$ y
$S=[s_1,s_2\ldots,s_n]$ dos permutaciones de $T$. Donde $R$ es la
salida que produciría nuestro algoritmo y $S$ es la solución óptima.
Sabemos que $R$ cumple $r_1 \leq r_2 \leq \ldots \leq r_n$, probemos
que $S$ debe cumplir lo mismo.

La suma de los tiempos de espera de $S$ es menor o igual que la de
cualquier otra solución. Por tanto \(C_s = \sum_{i=1}^n (n+1-i)s_i\)
es mínima. Supongamos que $S$ no estuviera ordenada en orden no
decreciente, esto es
\[\exists p,q \in \{1,2,\ldots,n\} \mbox{ tales que } p<q \mbox{ y } s_p
  > s_q\]
Tendremos entonces
\begin{align*}
  s_p > s_q& \qquad n+1-q > n+1-p\\
  s_p > s_q& \qquad ((n+1-q) - (n+q-p)) > 0 \\
\end{align*}

\vspace{-15mm}

\begin{align*}    
  ((n+1-q) - (n+q-p))s_p > ((n+1-q) - (n+q-p))s_q \\
  (n+1-q)s_p - (n+1-p)s_p > (n+1-q)s_q - (n+1-p)s_q
\end{align*}
y por tanto
\[(n+1-q)s_p + (n+1-p)s_q > (n+1-q)s_q + (n+1-p)s_p\]

Esto significa que intercambiar las posiciones de los tiempos $s_p$ y
$s_q$ en la lista $S$ produciría una suma de tiempos menor a la
existente, lo que contradice la hipótesis de que $S$ es una solución
óptima. Tenemos entonces $s_1 \leq s_2 \leq \ldots \leq s_n$, luego
$S=R$ como queríamos.

\hfill\qedsymbol
\end{document}
